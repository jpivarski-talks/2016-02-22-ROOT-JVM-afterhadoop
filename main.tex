\documentclass{beamer}

\mode<presentation>
{
  \usetheme{default}      % or try Darmstadt, Madrid, Warsaw, ...
  \usecolortheme{default} % or try albatross, beaver, crane, ...
  \usefonttheme{default}  % or try serif, structurebold, ...
  \setbeamertemplate{navigation symbols}{}
  \setbeamertemplate{caption}[numbered]
  \setbeamertemplate{footline}[frame number]
} 

\hypersetup{colorlinks=true, urlcolor=blue}

\usepackage[english]{babel}
\usepackage[utf8x]{inputenc}
\usepackage{dirtree}
\usepackage{listings}
\usepackage{courier}

\title[2016-02-22-ROOT-JVM-afterhadoop]{Accessing ROOT from the JVM (update)}
\author{Jim Pivarski}
\date{2016-02-22}

\xdefinecolor{darkblue}{rgb}{0.1,0.1,0.7}
\definecolor{mygreen}{rgb}{0,0.6,0}
\definecolor{mygray}{rgb}{0.5,0.5,0.5}
\definecolor{mymauve}{rgb}{0.58,0,0.82}

\lstset{ %
  backgroundcolor=\color{white},   % choose the background color
  basicstyle=\ttfamily\scriptsize,        % size of fonts used for the code
  breaklines=true,                 % automatic line breaking only at whitespace
  captionpos=b,                    % sets the caption-position to bottom
  commentstyle=\color{mygreen},    % comment style
  escapeinside={\%*}{*)},          % if you want to add LaTeX within your code
  keywordstyle=\color{blue},       % keyword style
  stringstyle=\color{mymauve},     % string literal style
}

\begin{document}

\begin{frame}
  \titlepage
\end{frame}

% Uncomment these lines for an automatically generated outline.
%\begin{frame}{Outline}
%  \tableofcontents
%\end{frame}

\begin{frame}{Motivation (reminder)}
\begin{block}{}
\vspace{-\baselineskip}
Most of the big data-pipeline frameworks used in industry run on the Java Virtual Machine (JVM) and most physics data is in ROOT, so we need a bridge.
\end{block}

\begin{block}{}
\vspace{-\baselineskip}
Target use-cases:
\begin{itemize}
\item Access ROOT TTrees (only) in various Apache data pipeline tools (Hadoop, Spark, Storm\ldots)
\item Allow physicists to skim/slim their group's analysis ntuples in Spark.
\begin{itemize}
\item Potentially faster for iterative studies (skim, fix bug, reskim) because intermediate datasets can be cached in-memory.
\item Abstracts away file locations and transfers, focuses on data transformations.
\item Consolidates ad-hoc shell scripts to a single, programmable workflow.
\item Tree of {\tt map/filter/reduce} operations can simplify scanning (parameter scans, cut scans, \ldots).
\item May require some training to help physicists adopt the new paradigm.
\end{itemize}



\end{itemize}
\end{block}


\end{frame}

\end{document}
